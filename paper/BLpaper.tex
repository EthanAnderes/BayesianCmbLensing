\documentclass[noinfoline]{imsart}


\usepackage{bm}
\usepackage{graphics,epsfig,rotate,lscape,graphicx,amsmath,amsthm,amssymb,float,amsfonts,amsbsy,hyperref,delarray,sectsty,amsfonts,amscd,pifont}
\usepackage{color,multirow}
\usepackage{algorithmicx}
\usepackage{algorithm}
\usepackage{enumerate}
\usepackage{microtype}
\usepackage{wrapfig}

\usepackage{geometry}
\geometry{letterpaper,left=1.2in,right=1.2in,top=1.2in,bottom=1.1in}

\bibliographystyle{plain}


\newtheorem{claim}{Claim}
\newtheorem{proposition}{Proposition}
\newtheorem{theorem}{Theorem}
\newtheorem{lemma}{Lemma}
\newtheorem{conjecture}{Conjecture}
\newtheorem{corollary}[theorem]{Corollary}
\newtheorem{definition}{Definition}
\newtheorem{notation}{Notation}
\renewcommand{\labelenumi}{(\roman{enumi})}
\newcommand{\bx}{\boldsymbol x}
\newcommand{\by}{\boldsymbol y}
\newcommand{\bl}{\boldsymbol l}
\newcommand{\bk}{\boldsymbol k}
\newcommand{\re}{\text{\,\rm re}}
\newcommand{\im}{\text{\,\rm im}}
\newcommand{\bs}{\boldsymbol}



% --------------------------------------
\begin{document}



\begin{frontmatter}
\title{Bayesian inference of CMB gravitational lensing}
\runtitle{Bayesian CMB lensing}
\begin{aug}
   \author{\fnms{Ethan}  \snm{Anderes}\corref{}\thanksref{t1}\thanksref{a}}
   \author{\fnms{Benjamin} \snm{Wandelt}\thanksref{b}\thanksref{c}\thanksref{d}}
   \and
   \author{\fnms{Guilhem}  \snm{Lavaux}\thanksref{b}\thanksref{c}}%
   \ead[label=u1, url]{https://github.com/EthanAnderes/BayesianCmbLensing.git}
   \address[a]{Department of Statistics, University of California, Davis CA 95616, USA.}
   \address[b]{
   Sorbonne Universités, UPMC Univ Paris 06, UMR7095, Institut
  d’Astrophysique de Paris, F-75014, Paris, France}
  \address[c]{CNRS, UMR7095, Institut d’Astrophysique de Paris, F-75014, Paris, France}
  \address[d]{Chaire d'excellence, Lagrange Institute (ILP) 98 bis, boulevard Arago 75014 Paris France}
   \thankstext{t1}{Research supported by: NSF DMS-1007480, NSF CAREER DMS-1252795}
   \thankstext{t2}{Code available at \printead{u1}}
   \runauthor{E. Anderes et al.}
\end{aug}

\begin{abstract} 
The Planck satellite, along with ground based telescopes such as the Atacama Cosmology Telescope (ACT)  and  the South Pole Telescope (SPT), have mapped the  cosmic microwave background (CMB) at such an unprecedented resolution as to allow a detection of the subtle distortions  due to the gravitational influence of intervening  dark matter. This distortion is called  gravitational lensing and has become a  powerful probe of cosmology and dark matter. Estimating gravitational lensing  is important for two reasons. First, the weak lensing estimates can be used to construct a map of dark matter which would be invisible otherwise. Second,  weak lensing estimates can, in principle, un-lense the observed CMB to construct the original CMB radiation fluctuations. Both of these maps,  the unlensed CMB radiation field and the dark matter field, are deep probes of cosmology and cosmic structure. Bayesian techniques seem a perfect fit for the statistical analysis of lensing and the CMB. One reason is that the priors for the unlensed CMB and the lensing potential are---very nearly---Gaussian random fields. The Gaussianity coming from physically predicted  quantum randomness in the early universe. However, challenges associated with a full Bayesian analysis have prevented previous attempts at  developing a working Bayesian prototype. 
In this paper we solve many of these obstacles with a re-parameterization of CMB lensing. This  allows us to obtain draws from the  Bayesian lensing posterior of both the lensing potential and the unlensed CMB which converges remarkably fast. 
\end{abstract}

\begin{keyword}
\kwd{CMB}
\kwd{gravitational lensing}
\kwd{Bayesian}
\kwd{Gibbs sampler}
\kwd{Ancillary Gibbs chain}
\kwd{Sufficient Gibbs chain}
\end{keyword}

\end{frontmatter}





 Over the past few years, data from ground based telescopes (ACT, SPT and Bicep2) and the Planck satellite and have resulted in an unprecedented detection of weak gravitational lensing of the cosmic microwave background (CMB) \cite{das2011detection,van2012measurement,ade2013planck}.  This new data not only probes the nature of dark matter but also constrains cosmological models of gravity waves and dark energy. The state-of-the-art estimator of CMB gravitational lensing, the quadratic estimator developed by Hu and Okomoto \cite{hu2001mapping,hu2002mass}, works in part through a delicate cancellation of terms in an infinite Taylor expansion of the lensing effect on the CMB. The effect of this cancellation is particularly sensitive to foreground contaminants and sky masking, which  if not fully accounted for,  limits  the statistical inferential power of this new data.  

 Possibly the most promising alternative to the quadratic estimator is Bayesian lensing. Indeed, Bayesian techniques applied to the lensed CMB observations have the potential for drastically changing the way lensing is estimated and used for inference.  Current frequentist estimators of the unknown lensing potential treat the unlensed CMB as a source of shape noise which is marginalized out. Conversely, a Bayesian lensing posterior treats the lensing potential {\em and} the unlensed CMB as joint unknowns, whereby obtaining scientific constrains jointly rather than marginally. Moreover, the  posterior distribution is easier to interpret and sequentially update with additional data. From the geometry of weak lensing, most of the lensing power comes from matter at high redshift. At these distances the matter distribution on large scales is well approximated by Gaussian density fluctuations. In addition, the unlensed CMB is, at present, indistinguishable from an isotropic Gaussian random field.  From a statistical perspective, this is a perfect scenario for Bayesian methods in that both the observations and the unknown lensing potential are {\em physically predicted} to be Gaussian random fields.  
 
Physicists have known, for some time, that Bayesian methods could potentially provide next-generation lensing estimates. In their seminal review \cite{Lewis20061}, authors Lewis and Challinor  discuss the possibility of obtaining posterior draws from the lensing potential and the unlensed CMB jointly. However, they acknowledge the main obstacle for naive Gibbs implementations:
\begin{quote}
``... given a particular lensing potential the delensed sky is given essentially by a delta function.
This means that naive Gibbs iterations will not converge within a reasonable time. At the time of writing there are no known practical methods for sampling from the full posterior distribution."
\end{quote}
In this paper we show that, indeed, there does exist a practical way to obtain Gibbs iterations which converge quickly. The solution is through a re-parameterization of CMB lensing problem. Instead of treating the lensing potential as unknown we work with inverse-lensing or what we call anti-lensing. Surprisingly, the slowness of naive Gibbs translates to fast convergence of the re-parameterized Gibbs chain. 



In Section \ref{two parameter system} we motivate our re-parameterization by analyzing a simple two parameter statistical problem.  The concepts are then applied to the Bayesian lensing problem in Section \ref{Section: Ancillary and sufficient parameters for the lensed CMB}. The two conditional distributions in our Gibbs implementation are discussed in Section \ref{Section: hamiltonian sampler section} and Section \ref{Section: iterative message passing section}. We finish with some  simulation examples in Section \ref{Section: simulation examples}.




%%%%%%%%%%%%%%%%%%%%%%%%%%%%%%%%%%%%%%%%%%%%
%
% Section
%
%%%%%%%%%%%%%%%%%%%%%%%%%%%%%%%%%%%%%%%%%%%%%%
\section{Weak lensing primer and a Bayesian challenge}
\label{primer}

The effect of weak lensing is to simply remap the CMB, preserving surface brightness.   Up to leading order, the remapping  displacements are given by $\nabla \phi$, where $\phi$ denotes the lensing potential and is the planar projection of the three dimensional gravitational potential (see Dodelson, S. \cite{dodelson2003modern}, for example). Therefore the lensed CMB can  be written $T(x + \nabla \phi(x))$ where $T(x)$ denotes the unlensed CMB temperature fluctuations and $x$ represents an observational direction on the unit sphere. For this paper we will be focusing on the small angle limit  so that $x$ is assumed to vary in a small patch of $ \Bbb R^2$. However, do not expect the fast convergence properties of our algorithm to be sensitive to the small angle approximation and the methodology presented here should hold for a full treatment on the sphere.  The lensed CMB is observed with additive noise (denoted $n(x)$) to result in data of the form
\begin{equation}
\label{data model}
 \text{data}(x)= T(x + \nabla \phi(x))+ n(x).
\end{equation}
The goal of weak lensing surveys is to use the data in (\ref{data model}) to  estimate $\phi$, $T$  and possibly  the spectral densities of $T$ and $\phi$.  

A natural approach to develop a Bayesian lensing estimator is to generate posterior samples through a Gibbs algorithm which iteratively samples from the two conditionals: $P(T |  \phi,\text{data})$ and $P(\phi | T,  \text{data})$.
Sampling from $P(T |  \phi,\text{data})$ is simply a Gaussian random field prediction problem since conditioning on $\phi$ models the data as
\[
\text{data}(x) = T(\!\!\!\underbrace{x+\nabla\phi(x)}_\text{\tiny known obs locations}\!\!\!) + n(x).
\]
In other words, the data is a noisy version of  $T$ observed on an irregular grid. 
Conversely, when sampling from $P(\phi | T,\text{data})$ the data is of the form
\[
\text{data}(x) = \underbrace{\!\!T\!\!}_{\text{\tiny known}}(x+\nabla\phi(x)) + n(x). 
\]
To see how one might approximate this conditional notice first that the CMB field $T(x)$ is very smooth. Indeed, Silk damping  predicts  a exponentially decaying power spectrum. Therefore a linear Taylor approximation,  $\text{data}(x) \approx T(x) + \nabla T(x)\cdot \nabla\phi(x) + n(x)$, may be useful. In fact, the derivation of the quadratic estimator explicitly uses this linear approximation. 
If one is willing to use this linear approximation then the conditional $P(\phi | T,\text{data})$ is simply a Bayesian regression problem since $T$ (and thus $\nabla T$) are both known with a Gaussian prior on $\nabla \phi$.

Unfortunately, the structure of both of these conditionals make the Gibbs very slow to converge. The case is exacerbated in the situation when noise level is small. For example, in the second conditional, if $T$ is known and fixed, the extent of the likely $\phi$'s under $P(\phi|T,\text{data})$  is very small compared to the likely $\phi$'s under $P(\phi, T| \text{data})$. 
This suggests a highly dependent posterior $P(\phi, T| \text{data})$. 










%%%%%%%%%%%%%%%%%%%%%%%%%%%%%%%%%%%%%%%%%%%%
%
% Section
%
%%%%%%%%%%%%%%%%%%%%%%%%%%%%%%%%%%%%%%%%%%%%%%
\section{Two parameter analogy}
\label{two parameter system}


To motivate our solution to the Bayesian lensing problem we start with a simple two parameter statistical problem.  This system has two unknown parameters $ t, \varphi$ with a single data point given by
\[\text{data} =  t + \varphi + n\]
where $n$ denotes additive noise.  In the Bayesian setting, the posterior distribution is computed as 
\begin{equation}
\label{post1}
 P( t,\varphi|\text{data})\propto P(\text{data}| t, \varphi) P( t,\varphi) 
 \end{equation}
where $P(\text{data}| t, \varphi)$ denotes the likelihood of the data given  $ t, \varphi$ and $P( t,\varphi)$ denotes the  prior on $ t, \varphi$. 
The Gibbs sampler is a widely used algorithm for generating (asymptotic) samples from  $P( t, \varphi|\text{data})$. The algorithm generates a Markov chain of parameter values $( t^{1}, \varphi^{1}), ( t^{2}, \varphi^{2}),\ldots$ generated by iteratively sampling from the conditional distributions:
\begin{align*}
 t^{i+1} &\sim P( t|\varphi^{i},\text{data}) \\
\varphi^{i+1}   &\sim P(\varphi| t^{i+1},\text{data}).
\end{align*}
A useful heuristic for determining the convergence rate of a Gibbs chain is the extent to which the two parameters $ t$ and $\varphi$ are dependent in $P(t, \varphi|\text{data})$. A highly dependent posterior $P( t, \varphi|\text{data})$ leads to a slow Gibbs chain, near independence leads to a fast Gibbs chain. Indeed, exact independence gives a sample of the posterior after one Gibbs step.  A technique for accelerating the convergence of a Gibbs sampler is to find a  re-parameterization of $ t$ and $\varphi$ in a way which makes the posterior less dependent. In the remainder of this section we discuss a specific re-parameterization which, by analogy, can be applied to Bayesian lensing.


The relevant situation for Bayesian lensing is the case that $ t$ and $\varphi$ are highly negatively correlated in $P( t, \varphi|\text{data})$.  This motivates re-parameterizing $( t,\varphi)$ to $(\widetilde  t, \varphi)$ where $\widetilde  t \equiv  t + \varphi$ so that
\begin{align*}
\text{data} &= \widetilde  t + n.
\end{align*}
In the statistics literature,  $( t, \varphi)$ has been referred to as an {\bf ancillary parameterization} whereas $(\widetilde  t, \varphi)$ is referred to as a {\bf sufficient parameterization}.
We note that the terms {\em ancillary} and {\em sufficient } parameterization have been used interchangeably with the nomenclature {\em non-centered} and {\em centered } parameterizations, respectively, in the statistics literature \cite{bernardo2003non,gelfand1995efficient,papaspiliopoulos2008stability,papaspiliopoulos2007general,yu2011center}.  Figure  \ref{fastslowGibbs} illustrates the difference between an ancillary versus sufficient posterior distribution for our simple two parameter model. The left plot shows the posterior density contours for the ancillary parameterization $( t, \varphi)$, along with 40 steps of a Gibbs sampler.  Conversely, the right plot shows the posterior density contours for the sufficient chain $(\widetilde  t, \varphi)$ with 40 Gibbs steps. Notice that negative correlation  in the ancillary parameterization manifests in near independence for the sufficient chain.  Indeed, the slower the ancillary chain the faster the sufficient chain and vice-versa. 
\begin{figure}[H]
\label{fastslowGibbs}
\includegraphics[height=2.0in]{graphics/theta.pdf}\includegraphics[height=2.0in]{graphics/thetatilde.pdf}
\caption{{\em Left:} density contours of the {\bf ancillary} chain $P( t, \varphi|\text{data})$ with 40 steps of a Gibbs sampler. {\em Right:} density contours of the {\bf sufficient}  chain $P(\widetilde  t, \varphi|\text{data})$ with 40 steps of a Gibbs sampler.}
\end{figure}
 







	
%%%%%%%%%%%%%%%%%%%%%%%%%%%%%%%%%%%%%%%%%%%%
%
% Section
%
%%%%%%%%%%%%%%%%%%%%%%%%%%%%%%%%%%%%%%%%%%%%%%
\section{Ancillary versus sufficient parameters for the lensed CMB}
\label{Section: Ancillary and sufficient parameters for the lensed CMB}

The ancillary parameterization presented in the previous section is analogous to the lensed CMB problem as follows
\begin{align*}
  \text{data}(x) &= T(x+\nabla \phi(x)) + n(x)\quad\text{\bf\em analogous to}\quad
  \text{data} =  t +\varphi + n
\end{align*}
where the unlensed  CMB temperature field $T$ and the lensing potential $\phi$ are the two unknown parameters. As was discussed in Section \ref{primer} the Gibbs chain based on the ancillary parameters $T(x)$ and $\phi(x)$ is exceedingly slow.  This clearly motivates the following re-parameterization to sufficient parameters for the lensed CMB problem 
\begin{align*}
  \text{data}(x) &= \widetilde T(x) + n(x) \quad\text{\bf\em analogous to}\quad
  \text{data} =  \widetilde t + n
\end{align*}
where now $\widetilde T$ denotes the lensed CMB temperature field with no noise or beam.
The sufficient chain then proceeds as
\begin{align}
\label{suff 1} \widetilde T^{i+1}&\sim P(\widetilde T |  \phi^{i},\text{data}) \\
\label{suff 2} \phi^{i+1}&\sim P(\phi | \widetilde T^{i+1},  \text{data}).
\end{align}
In Section \ref{Section: iterative message passing section} we adapt an iterative message passing algorithm, originally developed in \cite{elsner2013efficient,jasche2014matrix}, for Wiener filtering and sampling from (\ref{suff 1}). In Section \ref{Section: hamiltonian sampler section} we derive a Hamiltonian Markov Chain algorithm to sample from (\ref{suff 2}). Our Hamiltonian Markov Chain algorithm relies on an approximation---motivated again by the two parameter system---we call {\em anti-lensing}.







%%%%%%%%%%%%%%%%%%%%%%%%%%%%%%%%%%%%%%%%%%%%
%
% Section
%
%%%%%%%%%%%%%%%%%%%%%%%%%%%%%%%%%%%%%%%%%%%%%%
\subsection{Anti-lensing approximation}
\label{section: Anti-lensing approximation}

In the two parameter analogy from Section \ref{two parameter system},  the relation between the sufficient parameter $\widetilde t$ and the ancillary parameter $t$ is given by $\widetilde t - \varphi = t$. The corresponding relation for CMB lensing we refer to as {\em anti-lensing}
\begin{equation}
\label{antilense}
 \widetilde T(\,\underbrace{x-\nabla\phi(x)}_\text{\small anti-lensing}\,) \approx T(x).% \quad\text{\bf\em analogous to}\quad \widetilde t - \varphi = t.
\end{equation}
We distinguish between {\em inverse lensing} and  {\em anti-lensing}. Inverse lensing denotes the true coordinate displacement which, when applied to  $\widetilde T$, recovers the unlensed $T$. Conversely,  anti-lensing is  given by $-\nabla\phi$ and approximates inverse lensing.  
To examine the difference between these two operations, start with a Helmholtz decomposition of the inverse lensing displacement: $-\nabla \phi^\text{inv}(x) - \nabla^\perp \psi^\text{inv}(x),$  where $\nabla^\perp \equiv \bigr(-\frac{\partial}{\partial y},\frac{\partial}{\partial x} \bigl)$ and  $\psi^\text{inv}$ denotes a stream function potential which models a field rotation. Now $\phi^\text{inv}$ and $\psi^\text{inv}$ are used to convert a lensed CMB $\widetilde T$ to $T$ as follows 
\begin{align*}
\widetilde T\bigl(\,\underbrace{x-\nabla \phi^\text{inv}(x) - \nabla^\perp \psi^\text{inv}(x)}_\text{\small inverse lensing}\,\bigr) &= T(x).
\end{align*}
Due to the fact that the expected size of the lensing displacement $\nabla\phi$ is much smaller than the correlation length scale of $\phi$ we have
\begin{align}
\label{anti approx}
-\nabla\phi &\approx -\nabla\phi^\text{inv} \approx -\nabla\phi^\text{inv}- \nabla^\perp \psi^\text{inv}.
\end{align}
Indeed, the typical displacement size $\nabla \phi(x)$ is less than 3 arcmin whereas the correlation length scale of $\phi$ is on the order of degrees. In Figure \ref{antilensing plots} we show a simulation of  $-\phi$ (upper left) with the corresponding inverse lensing potential $-\phi^\text{inv}$ (upper right). The difference $\phi - \phi^\text{inv}$ is also shown (bottom left) along with the stream function $-\psi^\text{inv}$. Clearly, the magnitude of the difference  $\phi^\text{inv}-\phi$ and $-\psi^\text{inv}$ is sub-dominant to estimation error expected in current lensing experimental conditions. 


\begin{figure}[t]
\label{antilensing plots}
\includegraphics[height=2.4in]{graphics/anti_plt1.pdf}\includegraphics[height=2.4in]{graphics/anti_plt2.pdf}\\%
\includegraphics[height=2.4in]{graphics/anti_plt3.pdf}\includegraphics[height=2.4in]{graphics/anti_plt4.pdf}
\caption{The difference between anti-lensing and inverse lensing. {\em Upper left:} anti-lensing potential $-\phi$. {\em Upper right:} The inverse lensing potential $-\phi^\text{inv}$. {\em Bottom left:} The difference $\phi^\text{inv}-\phi$. {\em Bottom right:} The inverse lensing stream function $-\psi^\text{inv}$.}
\end{figure}








%%%%%%%%%%%%%%%%%%%%%%%%%%%%%%%%%%%%%%%%%%%%
%
% Section
%
%%%%%%%%%%%%%%%%%%%%%%%%%%%%%%%%%%%%%%%%%%%%%%
\section{Hamiltonian Monte Carlo sampler for $P(\phi | \widetilde T,  \text{\rm data})$}
\label{Section: hamiltonian sampler section}


The Hamiltonian Monte Carlo (HMC) algorithm is an iterative sampling algorithm designed to mitigate the low-acceptance rate of the Metropolis-Hastings algorithm when working in high dimension. 
A nice review of HMC can be found in \cite{neal2011mcmc}. For applications of HMC in cosmology see \cite{PhysRevD.75.083525, taylor2008fast, elsner2010local, 2010MNRAS.409..355J, 2012MNRAS.425.1042J, jasche2013bayesian, jasche2013methods}. In the present case we utilize the HMC algorithm to produce samples of $\phi$ from $P(\phi | \widetilde T,  \text{\rm data})$.  The key to making HMC work for lensing is to parameterize  $\phi$ in terms of it's Fourier transform. One can then utilize  Claim \ref{grad claim}, presented below, to efficiently compute the gradient of the log conditional density of $P(\phi | \widetilde T,  \text{\rm data})$, which is a necessary computation for the HMC algorithm.

{\em Notation:} Throughout the remainder of this paper, the Fourier transform of any function $f(x)$ will be denoted by $f_l$ or $f_k$ so that $f_l  =  \int_{\Bbb R^2} e^{-i x\cdot l}  f(x)\frac{dx}{2\pi}$ and
$f(x) =  \int_{\Bbb R^2} e^{i x\cdot l}  f_l \frac{dl}{2\pi}$ 
where $l\in \Bbb R^2$ is a two dimensional frequency vector and $x\in \Bbb R^2$  is a two dimensional spatial coordinate.


To describe the HMC algorithm let $\bs \phi$ denote the concatenation of the real and imaginary parts of $\phi_l$ as $l$ ranges through discrete frequencies $l$ ranging up to a pre-specified $|l|_{max}$ (but excluding  half of the Fourier frequencies due to the Hermitian symmetry associated with the Fourier transform of a real field). Note that $\bs \phi$ is a vector of real numbers. 
Let $P(\bs \phi|\widetilde T, \text{data})$ denote the density of $\bs \phi$ given $\widetilde T$ and the data. 
Let $\bs p$ denote a `momentum' vector and $\bs m$ denote a `mass' vector, which are both the same length as $\bs \phi$. The Hamiltonian is a function of $\bs \phi$ and $\bs p$ and is defined as follows
\[ H(\bs \phi, \bs p):= -\log P(\bs \phi|\widetilde T, \text{data})+\sum_k \frac{\bs p_k^2}{2\bs m_k^2}. \]
This Hamiltonian generates a time-dependent evolution of $\bs \phi$ and $\bs p$ given by 
\begin{align*}
\frac{d\bs \phi^t}{dt} &= \phantom{-}\nabla_{\bs p} H(\bs \phi^t, \bs p^t) \\
\frac{d\bs p^t}{dt}    &= -\nabla_{\bs \phi} H(\bs \phi^t, \bs p^t).
\end{align*} 
The HMC is a discrete version of  this time-dynamic equation, using a leapfrog method, which produces a Markov chain $(\bs \phi_1, \bs p_1), (\bs \phi_2, \bs p_2), \ldots$ where the $i^\text{th}$ iteration is given by Algorithm \ref{ith step of HMC} below.

\begin{algorithm}[H]
\small
\caption{ $i^\text{th}$ step of the Hamiltonian Markov Chain}
\label{ith step of HMC}
\begin{algorithmic}[1]
\State Set $\bs \phi^0:= \bs \phi_{i-1}$ and simulate $\bs p^0 \sim \mathcal N(0,\Lambda_m^2 )$ where $\Lambda_m$ is diagonal with  $diag(\Lambda_m)=\bs m$.
\State  Recursively compute $\bs \phi^{k\epsilon}$ and $\bs p^{k\epsilon}$ for $k=1,\ldots, n$ using the following equations:
\begin{align*}
\bs \phi^{t+\epsilon} &:= \bs \phi^{t} + \epsilon \Lambda_m^{-1} \left[ \bs p^t - \frac{\epsilon}{2} \nabla_{\bs \phi} H(\bs \phi^t, \bs p^t) \right]. \\
\bs p^{t+\epsilon} &:= \bs p^{t} - \frac{\epsilon}{2}\Bigl[ \nabla_{\bs \phi} H(\bs \phi^t, \bs p^t) + \nabla_{\bs \phi} H(\bs \phi^{t+\epsilon}, \bs p^t)\Bigr]
\end{align*}
% \begin{align*}
% \bs p^{t+\epsilon/2} &:= \bs p^{t} + \frac{\epsilon}{2} \nabla_{\bs \phi} H(\bs \phi^t, \bs p^t) \\
% \bs \phi^{t+\epsilon} &:= \bs \phi^{t} + \frac{\epsilon}{2} \nabla_{\bs \phi} H(\bs \phi^t, \bs p^t) \\
% \bs p^{(k+2)\epsilon} &:= \bs p^{(k+2)\epsilon/2} - \frac{\epsilon}{2}\Bigl[ \nabla_{\bs \phi} H(\bs \phi^t, \bs p^t) + \nabla_{\bs \phi} H(\bs \phi^{t+\epsilon}, \bs p^t)\Bigr]
% \end{align*}
\State Simulate $u\sim \mathcal U(0,1)$, and define $p:= \min\left(1,{e^{- H(\bs \phi^{n\epsilon}, \bs p^{n\epsilon})}}/{e^{-H(\bs \phi^0, \bs p^0)}}\right)$.
\State  If $u< p$,  set $\bs \phi_{i}:= \bs \phi^{n\epsilon}$, otherwise set  $\bs \phi_{i}:= \bs \phi_{i-1}$.
\end{algorithmic}
\end{algorithm}

The HMC algorithm is notoriously sensitive to tuning parameters. 
The prevailing wisdom is that one should set $\bs m$ to match the posterior variance. For the problems given in the simulation simply set  $\bs m$ to be nearly proportional  to $C_l^{\phi\phi}$ (with slightly more attenuation at low wavenumber).  As is advocated in \cite{taylor2008fast}  we  use a random $\epsilon$ to avoid resonant frequencies.


The key computational difficulty in using algorithm \ref{ith step of HMC} is the computation of the $\nabla_{\bs \phi} H(\bs \phi^t, \bs p^t)$, or equivalently the computation of $\nabla_{\bs \phi} \log P(\bs \phi|\widetilde T, \text{data})$. The number of frequencies is extremely large and therefore, any slow computation of the gradients will present a serious bottleneck. 
The follow claim shows that the gradient of the log density of $P(\phi|\widetilde T, \text{data})$, with respect to the Fourier basis of $\phi$, can be computed quickly with Fourier and inverse Fourier transforms. This claim makes the HMC algorithm computationally feasible for CMB lensing.

\begin{claim}
\label{grad claim}
 Under the anti-lensing approximation (\ref{antilense}) for any nonzero frequency vector $l\equiv (l_1, l_2) \in \Bbb R^2$ 
\begin{equation}
\label{grad claim eq}
 \frac{\partial}{\partial \phi_l}\log P(\phi | \widetilde T,  \text{\rm data}) \propto -  \frac{\phi_l}{C^{\phi\phi}_{l}} -  \sum_{q=1,2} i  l_q \int_{\Bbb R^2} e^{-i x\cdot l} A^q(x)B(x) \frac{dx}{2\pi}  
\end{equation}
 where  $\phi_l = \re \phi_l + i \im \phi_l$, $\frac{\partial}{\partial \phi_l}\equiv \frac{\partial}{\partial\re \phi_l} + i \frac{\partial}{\partial\im\phi_l}$ and 
 \begin{align}
 B_l &\equiv \frac{1}{C_l^{TT}} \int e^{-i x\cdot l}  \widetilde T(x-\nabla \phi(x))\frac{dx}{2\pi} \\ 
 A^q(x) &\equiv \frac{\partial\widetilde T}{\partial x_q}\bigl(x-\nabla \phi(x)\bigr).
 \end{align}
\end{claim}


Notice the anti-lensing approximation (\ref{anti approx}) is used in both terms of the right hand side of (\ref{grad claim eq}). Clearly the term $ \sum_{q=1,2} i  l_q \int_{\Bbb R^2} e^{-i x\cdot l} A^q(x)B(x) \frac{dx}{2\pi}  $ uses the anti-lensing approximation. However, one can adjust this term  to incorporate the inverse stream and potential functions $\psi^\text{inv}$ and $\phi^\text{inv}$, discussed in Section \ref{section: Anti-lensing approximation}. The main advantage of anti-lensing, then, is in the Gaussian prior term $\phi_l/{C^{\phi\phi}_{l}}$. 
Indeed, without the anti-lensing approximation one can no longer use a Gaussian prior on $\phi^\text{inv}$. This follows since the inverse lensing displacement is a lensed version of $-\nabla \phi$, hence the non-Gaussianity.  Therefore the advantage of  the anti-lensing approximation (\ref{anti approx}) is that it greatly simplifies the gradient computation (\ref{grad claim eq}) and allows a Gaussian approximation for the prior distribution of inverse lensing. 

{\em Remark:} It is interesting to note that the gradient in Claim \ref{grad claim} evaluated at $\phi = 0$, equals the un-normalized quadratic estimate in the case of a noise free observation. Indeed, an approximate Newton step, using (\ref{grad claim eq}), is an accurate approximation to the regular quadratic estimate. This gives an indication that the parameterization $(\widetilde T, \phi)$ results in fast convergence of the corresponding Gibbs chain.



%%%%%%%%%%%%%%%%%%%%%%%%%%%%%%%%%%%%%%%%%%%%
%
% Section
%
%%%%%%%%%%%%%%%%%%%%%%%%%%%%%%%%%%%%%%%%%%%%%%
\section{Iterative message passing algorithm for $ P(\widetilde T |  \phi,\text{\rm data})$}
\label{Section: iterative message passing section}

There are two natural ways to model the lensed CMB $\widetilde T$. If one marginalizes out $\phi$, then $\widetilde T$ is modeled as a {\em non-Gaussian} but isotropic random field. Conversely, if one conditions on $\phi$ the field $\widetilde T$ is modeled as a {\em non-isotropic} but Gaussian random field. The latter case is relevant for sampling from $ P(\widetilde T |  \phi,\text{\rm data})$ which  is, therefore,  simply a Gaussian conditional simulation problem. Unfortunately, the non-isotropic (indeed, non-stationary) nature of the conditional distribution of $\widetilde T$ presents serious computational challenges. In what follows we utilize a new iterative algorithm developed by  Wandelt and Elsner \cite{elsner2013efficient} for Gaussian conditional expectation when the signal is diagonalized in harmonic space that the noise is diagonalized in pixel space. The method we present here is similar to the Gibbs sampling adaptation of Jasche and Lavaux \cite{jasche2014matrix}. 


Start by transforming each pixel location $x$ to it's pre-lensed location $x+\nabla \phi(x)$, while simultaneously preserving the data associated with that pixel. This  effectively de-lenses  $\text{data}(x)= T(x+\nabla \phi(x))+n(x)$ but produces observations on an irregular grid. In particular, one may switch to the lensed coordinates $y = x+\nabla\phi(x)$ so that
\begin{align}
\nonumber
\underbrace{(x +\nabla \phi(x),\, \text{ data}(x))}_{\text{(pixel, data) tuple}} & = (y, \, T(y) + \tilde n(y)) 
\end{align}
where $\tilde n(x+\nabla\phi(x))=n(x)$.  Now the data $(y, \, T(y) + \tilde n(y))$ is arranged on an irregular grid in $y$. This irregular grid is then embedded into a high resolution regular grid by nearest neighbor interpolation. The points $y$ which do not get assigned an observation $T(y)+\tilde n(y)$ under the interpolation we consider to be masked. 
Figure \ref{embed} illustrates this situation. The left hand plot shows the irregularly sampled data $(x +\nabla \phi(x),\, \text{ data}(x))$ and the right hand plot shows the grid embedding. The filled dots represent observations of $T(y) + \tilde n(y)$ whereas the empty dots correspond to a masked observation of $T(y)$. 
Finally we extend the definition of $\tilde n(y)$ to have infinite variance over the masked region, whereby producing data $T(y)+\tilde n(y)$ over a dense regular grid in $y$.




\begin{figure}[H]
%{\includegraphics[height=2in]{Graphics/embed1.eps}}%
{\includegraphics[height=2.5in]{Graphics/embed2.eps}}%
{\includegraphics[height=2.6in]{Graphics/embed3.eps}}
\caption{The embedding of the lensed grid into a high resolution regular grid. The open circles represent masking.}
\label{embed}
\end{figure}


As a intermediate step in producing a sample from  $P(\widetilde T |  \phi,\text{\rm data})$ we produce a conditional  sample of $T(y)$ given the observations $T(y) + \tilde n(y)$. The difficulty of this step is that $\tilde n(y)$ is non-homogeneous noise---from the masking and any inhomogeneity in $n(x)$---and therefore it is not decorrelated by the Fourier transform. To handle this situation we adapted a new method developed by Wandelt and Elsner for Gaussian conditional expectation \cite{elsner2013efficient}. This method works particularly well for observations with large amounts of irregular masking, as in our case. The algorithm utilizes a messenger field which effectively behaves as a latent---signal plus white noise---model which is amenable to Gibbs sampling \cite{jasche2014matrix}.



\begin{algorithm}[H]
\small
\caption{Iterative message passing for sampling from $P(T | T + \tilde n)$}
\label{IMP}
\begin{algorithmic}[1]
\State Set cooling schedule $\lambda^1\geq  \ldots \geq \lambda^n = 1$. 
\State Simulate $Z^{1}(y), \ldots, Z^{n}(y)$ as independent realizations of standard Gaussian noise, in the pixel domain, so that $E \bigl[ Z^{k}(y)Z^{k}(y^\prime)\bigr] = \delta_{yy^\prime}$.
\State Simulate $W^{1}_l, \ldots, W^{n}_l$ as independent realizations of complex Gaussian white noise, in the Fourier domain, so that $E\bigl[ W^{n}_l W^{n\,*}_{l^\prime}\bigr] = \delta_{l - l^\prime}$.
\State Initialize the fields $M^0(y)$ and $T^0(y)$  to be zero.
\State Decompose the noise variance $\text{var}(\tilde n(y))$ into a homogeneous part $\bar\sigma^2$ and a non-homogeneous part $\tilde\sigma_y^{2}$ so that $\text{var}(\tilde n(y)) = \bar\sigma^2 + \tilde\sigma_y^{2}$ (setting $\tilde \sigma_y^2 = \infty$ on all masked pixels $y$).
\State Recursively compute the fields $T^k, M^k$ for $k=1, \ldots, n$ as follows:
\begin{align}
M^{k}(y) &:= \bigl[T(y) + \tilde n(y) \bigr]\frac{\lambda^{k}\bar\sigma^{2}}{\lambda^{k} \bar\sigma^2 + \tilde \sigma_y^2} + T^{k-1}(y)\frac{\tilde\sigma_y^{2}}{\lambda^k\bar\sigma^2 + \tilde \sigma_y^2}  + Z^{k}(y) \frac{1}{\sqrt{1/(\lambda^k\bar\sigma^{2}) + 1/\tilde\sigma_y^{2}}} \\
T^{k}_l &:= M_l^{k}\frac{C^{TT}_l}{C^{TT}_l + dy \, \lambda^k\bar\sigma^2} + W^{k}_l
\frac{1}{\sqrt{1/C^{TT}_l + 1/(dy \lambda^k\bar\sigma^2)}} 
\end{align}
where  $dy$ denotes the pixel grid area. 
\end{algorithmic}
\end{algorithm}


The following algorithm describes how to use Algorithm \ref{IMP} to produce a sample from $ P(\widetilde T |  \phi,\text{\rm data})$.

\begin{algorithm}[H]
\small
\caption{Sampling from $P(\widetilde T |  \phi,\text{\rm data})$}
\label{after HMC}
\begin{algorithmic}[1]
\State Embedded the de-lensed pixel/data pairs  $(x+\nabla\phi(x), \text{data}(x))$, as illustrated in Figure \ref{embed}, into observations of the form $(y, T(y)+\tilde n(y))$ where $y$ ranges over a high resolution regular grid.
\State Use Algorithm \ref{IMP} to produce a sample $T\sim P(T\,|\, T+\tilde n)$
\State Return $\widetilde T(x) = T(x+\nabla\phi(x))$.
\end{algorithmic}
\end{algorithm}




{\em Remark:} At present, Algorithm \ref{after HMC} is designed for the situation when the pixels are sufficiently small compared to the magnitude of $\nabla \phi(x)$ and the noise is approximately white on these scales.  The reason is that one needs to be able to transform the data $(x+\nabla\phi(x), \text{data}(x))$ to $(y, T(y) + \tilde n(y))$  in such a way that $T$ is diagonalized in harmonic space and $\tilde n$ is diagonalized in pixel space. 







%%%%%%%%%%%%%%%%%%%%%%%%%%%%%%%%%%%%%%%%%%%%
%
% Section
%
%%%%%%%%%%%%%%%%%%%%%%%%%%%%%%%%%%%%%%%%%%%%%%
\section{Simulation example}
\label{Section: simulation examples}

In this section we present a simulation to illustrate the methodology presented above. The simulated lensing potential used in this section, shown at left in Figure \ref{phix fig}, is generated on a flat sky with periodic boundary conditions. The data, shown at left in Figure \ref{tilde fig}, is generated on 2 arcmin pixels with independent additive noise and masking. The noise level is set to $8.0$ $\mu K$ arcmin and the masking covers approximately 10\% of the pixels.  The parameters of the Bayesian lensing procedure are the Fourier modes of $\widetilde T$ and $\phi$. For the lensing potential we set $|l|_\text{max}$ to $~460$. 
For this patch size the scale-resolution in Fourier space  $\delta l = 21$ yields $1500$ unknown Fourier coefficients for $\phi_l$. The $|l|_\text{max}$ of $2700$ for the unlensed temperature $T$ is set in Algorithm \ref{IMP} and corresponds to half of the Nyquist limit at $2$ arcmin pixels. 


\begin{wrapfigure}{r}{0.5\textwidth}
{\includegraphics[height=2.0in]{fromParallel/cooling.pdf}}%
\caption{
Cooling schedule for the message passing algorithm. This schedule is applied every $100^\text{th}$ step in the Gibbs algorithm.
 }
\label{cooling}
\end{wrapfigure}

 

We ran 15 parallel Gibbs chains for a total of 2500 steps. Each chain was initially warmed up by replacing the HMC draws in the first $5$ iterations with a gradient ascent.  A burn-in of of approximately 550 runs were discarded and the remaining runs were thinned by 100. The result is a total of 300 posterior samples. The cooling schedule for the iterative message passing algorithm, which is a key tuning parameter,  was selected by numerical experimentation.  Most of the Gibbs iterations set the cooling terms $(\lambda^1,\ldots, \lambda^{400}) \equiv (1,\ldots, 1)$  in Algorithm \ref{IMP}. However, we did find it advantageous to periodically run a nontrivial $1000$-step cooling schedule every $100^\text{th}$ pass of the Gibbs algorithm. This nontrivial cooling schedule for $\lambda^k$ is plotted in Figure \ref{cooling}.




Figures \ref{tilde fig}, \ref{phix fig} and \ref{slice fig}  illustrate the posterior draws in pixel space. The posterior average $E(\widetilde T(x) | \text{data})$ is approximated by the average of 300 posterior draws and is shown in the right hand plot of Figure \ref{tilde fig}. The posterior average $E(\phi(x) |\text{data})$ is similarly approximated by the average of 300 draws and is shown in the middle hand plot of Figure \ref{phix fig}. On the right-hand side of  Figure \ref{phix fig} we show the quadratic estimate of $\phi$ for comparison. However, due to the difficulty of using the quadratic estimate in the presence of sky cuts, the quadratic estimate shown uses all of the data---including the pixels which are masked---in producing the estimate of $\phi$. In general, one can see good agreement with $E(\phi(x) | \text{data})$  and $\phi$. Indeed, the effect of masking is visually undetectable as compared to the quadratic estimate.  To get a better visualization of the individual draws from the posterior, Figure \ref{slice fig} shows a horizontal cross section of the posterior draws. The plot for $\widetilde T(x)$ is magnified near the masking region for better visual inspection.





\begin{figure}[h]
{\includegraphics[height=2.8in]{fromParallel/data.pdf}}%
{\includegraphics[height=2.8in]{fromParallel/tildex_est.pdf}}%
\caption{{\em Left:} Data. {\em Middle:} posterior mean $E(\widetilde T(x)|\text{data})$.
 }
\label{tilde fig}
\end{figure}


\begin{figure}[h]
{\includegraphics[height=2.1in]{fromParallel/phix.pdf}}%
{\includegraphics[height=2.1in]{fromParallel/phix_est.pdf}}%
{\includegraphics[height=2.1in]{fromParallel/qex.pdf}}%
\caption{{\em Left:} simulation truth $\phi(x)$. {\em Middle:} posterior mean $E(\phi(x)|\text{data})$. {\em Right: Quadratic estimate with additional masked data.}}
\label{phix fig}
\end{figure}


\begin{figure}[h]
{\includegraphics[height=2.5in]{fromParallel/phix_slice.pdf}}%
{\includegraphics[height=2.5in]{fromParallel/tildex_ave_slice.pdf}}
\caption{sliced posterior draws.}
\label{slice fig}
\end{figure}



In Figures \ref{p spec fig} and \ref{t spec fig} we summarize the posterior draws for $\phi$ and $\widetilde T$ in the Fourier domain. The left plot shows $95\%$ posterior regions for  $l^4|\phi_l|^2/(4\delta_0)$  averaged over $l$ in wavenumber bins. For comparison the simulation true values of  $l^4|\phi_l|^2/(4\delta_0)$ are shown in red and the spectral density $l^4 C^{\phi\phi}_l/4$ is plotted in the black solid line. In the right plot of Figure \ref{p spec fig} we show the empirical cross correlation of the posterior draws over wavenumber bins with the simulation truth. In particular, the correlation between $\phi^\text{true}_l$ and  $\phi_l$ over all $l$  in a fixed wavenumber bin, where $\phi^\text{true}$ denotes the simulation truth and $\phi_l$ denotes a posterior draw.  The corresponding plots for $\widetilde T_l$ are shown in Figure \ref{t spec fig}.




\begin{figure}[h]
{\includegraphics[height=2.5in]{fromParallel/specP.pdf}}%
{\includegraphics[height=2.5in]{fromParallel/corrP.pdf}}
\caption{{\em Left:} Spectral power estimates. {\em Right:} Empirical cross correlation with the simulation truth at each wavenumber bin.}
\label{p spec fig}
\end{figure}




\begin{figure}[h]
{\includegraphics[height=2.5in]{fromParallel/specT.pdf}}%
{\includegraphics[height=2.5in]{fromParallel/corrT.pdf}}
\caption{{\em Left:} Spectral power estimates. {\em Right:} Empirical cross correlation with the simulation truth at each wavenumber bin.}
\label{t spec fig}
\end{figure}






%%%%%%%%%%%%%%%%%%%%%%%%%%%%%%%%%%%%%%%%%%%%
%
% Section
%
%%%%%%%%%%%%%%%%%%%%%%%%%%%%%%%%%%%%%%%%%%%%%%
\section{Concluding remarks}

In this paper we construct a prototype algorithm which establishes that it is possible to construct a fast Gibbs sampler of the Bayesian posterior for the unknown lensing potential and the de-noised CMB temperature map. This prototype solves one of the fundamental obstacles in a Gibbs implementation of the Bayesian lensing: the na\"ive parameterization $(T, \phi)$ is extremely slow. We identify the ancillary and sufficient parametrization duality for this problem and notice that the slowness of the Gibbs chain for the ancillary parametrization  $(T, \phi)$ translates to a fast chain for the sufficient parametrization  $(\widetilde T, \phi)$. This observation is one of the main contributions of this paper. The second contribution is the use of the anti-lensing approximation along with Claim \ref{grad claim} which makes feasible the development of a Hamiltonian Markov Chain algorithm for sampling from  $P(\phi | \widetilde T)$. Without the Fourier transform characterization in Claim \ref{grad claim} the HMC would be computational prohibitive. The third contribution of this paper is to recognize that a new messenger algorithm \cite{elsner2013efficient,jasche2014matrix} can be adapted for high resolution conditional Gaussian sampling under the irregular sampling scenario needed for $P(\widetilde T|\phi, \text{data})$.  Finally all the code presented here is written in the language {\em Julia} \cite{bezanson2012julia}  and is publicly available through the on-line github repository.  

% In contrast to enumerating the contributions, as above, we also discuss what has not been done, open problems and future work. First, we emphasize that what we present here is a prototype which does not yet handle the full suite of complexities present in the state-of-the-art Planck data set. Indeed, although the high resolution embedding works very well for moderately sized surveys (with small beams), the high resolution approach becomes untenable in the Planck setting due to the fact that the large number of pixels ($\sim 10$ million) makes high resolution embedding difficult. Moreover, the size of the Planck beam, we conjecture, is too large for our de-lensing approach. 
% Another extension of our prototype, necessary for Planck scientific contributions, is to incorporate spectral density uncertainties and/or parameter estimation in the Gibbs algorithm.

Notice that both sampling steps $P(\phi |\widetilde T)$ and $P(\widetilde T|\phi, \text{data})$ in our algorithm utilize a high resolution embedding for $\widetilde T$. In this paragraph we discuss what is needed to avoid using this embedding for scaling up this algorithm. When sampling from the conditional $P(\phi |\widetilde T)$, the main challenge is to compute $A^q(x)$ and $B(x)$, as defined in Claim  \ref{grad claim}.  Within the HMC algorithm, a proposed lensing potential $\phi$ changes iteratively. A each iteration  one requires a new computation of $A^q(x)$ and $B(x)$. In our prototype, a spline interpolation performs the task of fast anti-lensing required for  $A^q(x)$ and $B(x)$. It is an open problem how to compute this fast anti-lensing without the need for a high resolution $\widetilde T$. Simulating from $P(\widetilde T|\phi, \text{data})$ also requires a high resolution embedding in our prototype. This simply expresses the fact that given $\phi$ the field $\widetilde T$ is modeled as a non-stationary random field. To circumvent this difficulty we  transform to lensed coordinates as illustrated in Figure \ref{embed}. The challenge when avoiding this high resolution embedding, then, is to directly generate conditional simulations of the non-stationary $\widetilde T$ given $\text{data}(x)=\widetilde T(x)+ n(x)$ and the lensing potential $\phi(x)$. The non-stationarity of $\widetilde T$ and the fact that this simulation must be repeated at each Gibbs iteration makes this an interesting challenge.

%%%%%%%%%%%%%%%%%%%%
%
%  acknowledgments
%
%%%%%%%%%%%%%%%%%%%%%%

\section*{Acknowledgments}
BW acknowledges funding through his Chaire dÕExcellence from the Agence Nationale
de la Recherche (ANR-10-CEXC-004-01). This work has been done within the Labex ILP (reference ANR-10-LABX-63) part of the Idex SUPER, and received financial state aid managed by the Agence Nationale de la Recherche, as part of the programme Investissements d'avenir under the reference ANR-11-IDEX-0004-02.

\bibliography{refs}



%%%%%%%%%%%%%%%%%%%%
%
%  appendix
%
%%%%%%%%%%%%%%%%%%%%%%
\appendix

\section{}

Before we proceed to the proofs we say a few words regarding notation.
First, we do not differentiate, notationally, from the case of smooth random field with periodic boundary conditions defined on $(-L/2, L/2]^2$ and the case where $L\rightarrow \infty$ so that the Fourier series $\sum_{l \in \frac{2\pi}{L}\Bbb Z }   e^{i x\cdot l}  f_l \frac{2\pi/ L}{2\pi} $ converges to the continuous Fourier transform $\int_{\Bbb R^2}  e^{i x\cdot l}  f_l \frac{dl}{2\pi} $. %Indeed,  $\int_{\Bbb R^2}  e^{i x\cdot l}  f_l \frac{dl}{2\pi} $ is simply equivalent notation for the Riemann sum, with $L$ near infinity.
For example, at times we will refer to an infinitesimal area element $dl$ or $dk$ in Fourier space, which simply equals $(2\pi / L)^2$ for large $L$. In this case $\delta_l$ denotes a discrete dirac delta function which we equate with $1/dk$ when $l=0$ and zero otherwise. 
Secondly, for any function $f(x)$ let $f^\phi(x) = f(x-\nabla \phi(x))$ denote anti-lensing of $f$ and $f^\phi_l$ denote the Fourier transform of  $f^\phi(x)$.





\begin{proof}[{ Proof of Claim \ref{grad claim}}]
Since $\widetilde T$ is sufficient for the unknown $\phi$ we have that 
\begin{align*}
P(\phi|\widetilde T, \text{\rm data}) &= P(\phi|\widetilde T)\propto P(\widetilde T|\phi)P(\phi).
\end{align*}
Since $\phi(x)$ is an isotropic random field with spectral density $C_l^{\phi\phi}$ we have that $E(\phi^{\phantom{*}}_l \phi_{l^\prime}^*) = \delta_{l - l^\prime}C_l^{\phi\phi}$. 
Therefore  $E(\phi_l^{\phantom{*}}\phi_l^*) = \delta_{0}C_l^{\phi\phi}$ and $E(\phi_l\phi_{l}) = 0$ implies that the random variables $\re \phi_l$, $\im \phi_l$ are independent $\mathcal N(0, \frac{1}{2}\delta_0 C_l^{\phi\phi})$ for each fixed $l$.
Moreovoer  $\phi(x)$ takes values in $\Bbb R$ so that  $\phi_l = \phi_{-l}^*$.  This implies 
that $\phi_l$ and are independent random variables over all $l$ which are restricted to the Hermitian half of the Fourier grid, denoted $\Bbb H$ here. In particular,  if we exclude the zero frequency $l = 0$ we get
\begin{align}
\log P(\phi) - c_1 &=  - \frac{1}{2}\sum_{k \in \Bbb H\setminus \{0 \}}  \left[\frac{(\re \phi_k)^2}{\frac{1}{2}\delta_0 C_k^{\phi\phi}} +  \frac{(\im \phi_k)^2}{\frac{1}{2}\delta_0 C_k^{\phi\phi}} \right] = - \frac{1}{2}\int_{\Bbb R^2} \frac{|\phi_k|^2}{C_k^{\phi\phi}} dk \label{dr1} \\
\log P(\widetilde T| \phi) - c_2 &=  - \frac{1}{2}\sum_{k \in \Bbb H\setminus \{0 \}}  \left[\frac{(\re \widetilde T_k^\phi)^2}{\frac{1}{2}\delta_0 C_k^{TT}} +  \frac{(\im \widetilde T_k^\phi)^2}{\frac{1}{2}\delta_0 C_k^{TT}} \right] = - \frac{1}{2}\int_{\Bbb R^2} \frac{\bigl|\widetilde T_k^\phi\bigr|^2}{C_k^{TT}} dk \label{dr2}
\end{align}
where $c_1$ and $c_2$ are constants and  $\widetilde T^\phi(x)\equiv \widetilde T(x-\nabla \phi(x))$.
% -------- extra detail
% The integral approximation then becomes
% \[ \log P(\phi) = \text{constant} - \pi \int \left[\frac{(\re \phi_l)^2}{C_l^{\phi\phi}} +  \frac{(\im \phi_l)^2}{ C_l^{\phi\phi}} \right] \frac{dl}{2\pi} \]
% -------- extra detail
% Therefore
% \begin{align*}
% \frac{\partial}{\partial \re \phi_l}\log P(\phi)  &= -2 \frac{\re \phi_l}{\delta_0 C_l^{\phi\phi}} \\
% \frac{\partial}{\partial \im \phi_l}\log P(\phi)  &= -2 \frac{\im \phi_l}{\delta_0 C_l^{\phi\phi}}
% \end{align*}

{\em Remark:}
In the calculation  of (\ref{dr2}) we use the approximation that the lensing and anti-lensing operator (which is a linear action on the CMB) can be represented as an infinitesimal permutation matrix. This allows us to conclude that the determinant of the anti-lensing operator, $\det( d \widetilde T^\phi / d\widetilde T)$, is one. Therefore, by transformation of variables,   $P(\widetilde T|\phi) = P(\widetilde T^\phi|\phi) |\det( d \widetilde T^\phi / d\widetilde T)| = P(\widetilde T^\phi|\phi)$ which gives the right hand side of (\ref{dr2}) .


Taking derivatives in (\ref{dr1}) gives
\begin{equation}
\label{grad of prior}
\frac{\partial}{\partial \phi_l}\log P(\phi) = - 2(dl) \frac{\phi_l}{C^{\phi\phi}_{l}}.
\end{equation}
Taking derivatives in (\ref{dr2}) gives
\begin{align}
\frac{\partial}{\partial \re\phi_l}\log P(\widetilde T| \phi) 
% <----- extra detail
%&=  - dk\sum_{k \in \Bbb H} 2 \re \Bigl( \frac{\partial \widetilde T_k^\phi}{\partial \re\phi_l} \, \frac{\widetilde T_k^{\phi^*}}{C_k^{TT}} \Bigr)\\
&= - \re \int_{\Bbb R^2} \frac{\partial \widetilde T_k^{\phi}}{\partial \re\phi_l} \, \frac{\widetilde T_k^{\phi^*}}{C_k^{TT}}  \,dk \label{ll1}\\
\frac{\partial}{\partial \im\phi_l}\log P(\widetilde T| \phi) 
% <----- extra detail
%&=  - dk\sum_{k \in \Bbb H}2 \re \Bigl( \frac{\partial \widetilde T_k^\phi}{\partial \im\phi_l} \, \frac{\widetilde T_k^{\phi^*}}{C_k^{TT}} \Bigr) \\
&=  -\re \int_{\Bbb R^2}\frac{\partial \widetilde T_k^{\phi}}{\partial \im\phi_l} \, \frac{\widetilde T_k^{\phi^*}}{C_k^{TT}} \, dk \label{ll2}.
\end{align}
Taking linear combinations of the two equalities in Lemma \ref{partialconj} below we get
\begin{align}
\frac{\partial \widetilde T^\phi_k}{\partial \re \phi_{l}}  &= 
\frac{1}{2}\frac{\partial \widetilde T^\phi_k}{\partial\phi_{l}} + \frac{1}{2}\frac{\partial \widetilde T^\phi_k}{\partial\phi^*_{l}}
=
\frac{dk}{2 \pi}  \sum_{q=1,2} il_q\left\{ [(\nabla^q\widetilde T)^\phi]_{k-l}  -   [(\nabla^q\widetilde T)^\phi]_{k+l}  \right\}\\
\frac{\partial \widetilde T^\phi_k}{\partial \im \phi_{l} }  &= 
\frac{-i}{2}\frac{\partial \widetilde T^\phi_k}{\partial\phi_{l}} + \frac{i}{2}\frac{\partial \widetilde T^\phi_k}{\partial\phi^*_{l}}
=
\frac{ dk}{2 \pi} \sum_{q=1,2} l_q\left\{ -[(\nabla^q\widetilde T)^\phi]_{k-l}  -   [(\nabla^q\widetilde T)^\phi]_{k+l}  \right\}.
\end{align}
Now the above two equations establish, by Lemma \ref{forreal} below, that  both integrals $\int_{\Bbb R^2} \frac{\partial \widetilde T_k^\phi}{\partial \re\phi_l} \, \frac{\widetilde T_k^{\phi^*}}{C_k^{TT}}  \,dk$ and $\int_{\Bbb R^2}\frac{\partial \widetilde T_k^\phi}{\partial \im\phi_l} \, \frac{\widetilde T_k^{\phi^*}}{C_k^{TT}} \, dk$ are real
which implies
\begin{align*}
 \frac{\partial}{\partial \phi_l}\log P(\widetilde T| \phi) &= -\int_{\Bbb R^2} \frac{\partial \widetilde T_k^\phi}{\partial \phi_l} \, \frac{\widetilde T_k^{\phi^*}}{C_k^{TT}}  \,dk  \\
 &= -\frac{dk}{ \pi} \sum_{q=1,2} il_q \int_{\Bbb R^2}  [(\nabla^q\widetilde T)^\phi]_{k+l} \, \frac{\widetilde T_k^{\phi^*}}{C_k^{TT}}  \,dk \\
 &= -  i 2 (dk) \sum_{q=1,2} l_q \int_{\Bbb R^2}  [(\nabla^q\widetilde T)^\phi]_{k+l} \, \frac{\widetilde T_k^{\phi^*}}{C_k^{TT}}  \,\frac{dk}{2\pi} \\
 & = -  i 2 (dk) \sum_{q=1,2} l_q \int_{\Bbb R^2} e^{-i x\cdot l} A^q(x) \, B(x)  \,\frac{dx}{2\pi},\quad\text{by Lemma \ref{conv} below}
 \end{align*}
 where $A^q(x) \equiv (\nabla^q\widetilde T)^\phi(x)$ and $B_k\equiv (\widetilde T_k^{\phi})^* / C_k^{TT}$.
\end{proof}
% end proof



%%%%%%%%%%%%%%% lemma
\begin{lemma} 
\label{partialconj}
\begin{align}
\frac{\partial \widetilde T^\phi_k}{\partial \phi_{ l} }  &= \frac{  dk}{\pi}\sum_{q=1,2} \phantom{-}il_q[(\nabla^q\widetilde T)^\phi]_{k+l} \\
\frac{\partial \widetilde T^\phi_k}{\partial \phi^*_{ l} }  & = \frac{ dk }{\pi}\sum_{q=1,2} -il_q [(\nabla^q\widetilde T)^\phi]_{k-l}
\end{align}
where $\nabla^q \widetilde T\equiv \frac{\partial \widetilde T}{\partial x_q}$.
\end{lemma}
\begin{proof}
First notice
\begin{align}
\label{partial1}
\frac{\partial}{\partial \re \phi_{ l}}\frac{\partial\phi(x)}{ \partial x_q}  &= \int_{\Bbb R^2} i  k_q e^{ix \cdot k} \frac{\partial \phi_k}{\partial \re \phi_l }  \frac{d k}{2\pi} 
=\left[i  l_q e^{ix \cdot  l}  - i  l_q e^{-i x \cdot  l}  \right] \frac{d k}{2\pi}   \\
\label{partial2}
\frac{\partial}{\partial \im \phi_{ l}}\frac{\partial\phi(x)}{ \partial x_q}  &= \int_{\Bbb R^2} i  k_q e^{ix \cdot  k} \frac{\partial \phi_ k}{\partial \im \phi_{l} }  \frac{d k}{2\pi} 
=\left[-  l_q e^{ix \cdot  l}  -  l_q e^{-i x \cdot  l}  \right] \frac{d k}{2\pi}.  
\end{align}
This implies
\begin{align}
\nonumber \frac{\partial \widetilde T^\phi_k}{\partial \phi_{ l} } 
&=  \frac{\partial}{\partial  \phi_{ l} } \int_{\Bbb R^2}  e^{-ix \cdot k}\widetilde T(x-\nabla \phi(x))\frac{dx}{2\pi} \\
\nonumber &= \sum_{q=1,2}  \int_{\Bbb R^2} e^{-ix \cdot k} \nabla^q\widetilde T(x-\nabla \phi(x))\left[ -\frac{\partial}{\partial \re \phi_{ l}}\frac{\partial\phi(x)}{ \partial x_q} -i  \frac{\partial}{\partial \im \phi_{ l}}\frac{\partial\phi(x)}{ \partial x_q}\right]\frac{dx}{2\pi}\\
% %<-----extra detail
%  &= \frac{-d k}{2\pi} \sum_{q=1,2}  \int_{\Bbb R^2} e^{-ix \cdot k} \nabla^q\widetilde T(x+\nabla \phi(x))\left[ i  l_q e^{ix \cdot  l}  - i  l_q e^{-i x \cdot  l}  + i(-  l_q e^{ix \cdot  l}  -  l_q e^{-i x \cdot  l})  \right]\frac{dx}{2\pi} \\
% &= \frac{-d k}{2\pi}  \sum_{q=1,2} i l_q \int_{\Bbb R^2} e^{-ix \cdot k} \nabla^q\widetilde T(x+\nabla \phi(x))\left[ e^{ix \cdot  l}  -  e^{-i x \cdot  l}  -  e^{ix \cdot  l}  -  e^{-i x \cdot  l} \right]\frac{dx}{2\pi}\\
% %<-----
\nonumber &= \sum_{q=1,2}\frac{ il_q  dk}{\pi}  \int_{\Bbb R^2} e^{- ix \cdot (k+l)} \nabla^q\widetilde T(x-\nabla \phi(x))\frac{dx}{2\pi},\quad\text{by (\ref{partial1}) and (\ref{partial2})} \\
&=\sum_{q=1,2}\frac{  il_q dk}{\pi} [(\nabla^q\widetilde T)^\phi]_{k+l}
\end{align}
Similarly 
\begin{align}
\frac{\partial \widetilde T^\phi_k }{\partial \phi^*_l } 
% % <-----extra detail
% &= \frac{-d k}{2\pi}\sum_{q=1,2}  \int_{\Bbb R^2} e^{-ix \cdot k} \nabla^q\widetilde T(x-\nabla \phi(x))\left[ i  l_q e^{ix \cdot  l}  - i  l_q e^{-i x \cdot  l}  - i(-  l_q e^{ix \cdot  l}  -  l_q e^{-i x \cdot  l})  \right]\frac{dx}{2\pi} \\
% &= \frac{-d k}{2\pi} \sum_{q=1,2}  \int_{\Bbb R^2} e^{-ix \cdot k} \nabla^q\widetilde T(x-\nabla \phi(x))\left[ i  l_q e^{ix \cdot  l}  - i  l_q e^{-i x \cdot  l}  + i  l_q e^{ix \cdot  l}  +i  l_q e^{-i x \cdot  l}  \right]\frac{dx}{\pi} \\
% &= \sum_{q=1,2}  \frac{-i l_qd k}{\pi}\int_{\Bbb R^2} e^{-ix \cdot (k-l)} \nabla^q\widetilde T(x-\nabla \phi(x))\frac{dx}{\pi} \\
% % <-------
 &=\sum_{q=1,2}\frac{ -il_q dk}{\pi} [(\nabla^q\widetilde T)^\phi]_{k-l}.
\end{align}
\end{proof}





%%%%%%%%%%%% lemma
\begin{lemma} 
\label{forreal}
If $A(x)$ and $B(x)$ are real scalar fields then  the two  integrals,  $\int_{\Bbb R^2} i\bigl\{ A_{k-l}  -   A_{k+l}  \bigr\} B^*_k dk$ and  $\int_{\Bbb R^2}\bigl\{  A_{k-l}  +    A_{k+l}   \bigr\} B^*_k dk$, are both real numbers.
\end{lemma}


\begin{proof}
By a simple change of variables it is clear that 
$\int_{\Bbb R^2} \left(i\bigl\{ A_{k-l}  -   A_{k+l}  \bigr\} B^*_k\right)^* dk = \int_{\Bbb R^2} i\bigl\{ A_{k^\prime-l} - A_{k^\prime+l}  \bigr\} B_{k^\prime}^* dk^\prime$ and $\int_{\Bbb R^2} \left(\bigl\{  A_{k-l}  +    A_{k+l}   \bigr\} B^*_k \right)^* dk  = \int_{\Bbb R^2} \bigl\{ A_{k^\prime-l} + A_{k^\prime+l}  \bigr\} B_{k^\prime}^* dk^\prime$.

% \begin{align*}
% \int_{\Bbb R^2} \left(i\bigl\{ A_{k-l}  -   A_{k+l}  \bigr\} B^*_k\right)^* dk &= \int_{\Bbb R^2} -i\bigl\{ A_{-k+l}  -   A_{-k-l}  \bigr\} B_{k} dk \\
% &= \int_{\Bbb R^2} -i\bigl\{ A_{k^\prime+l}  -   A_{k^\prime-l}  \bigr\} B_{-k^\prime} dk^\prime \\
% &= \int_{\Bbb R^2} i\bigl\{ A_{k^\prime-l} - A_{k^\prime+l}  \bigr\} B_{k^\prime}^* dk^\prime
% \end{align*}

% \begin{align*}
% \int_{\Bbb R^2} \left(\bigl\{  A_{k-l}  +    A_{k+l}   \bigr\} B^*_k \right)^* dk &= \int_{\Bbb R^2} \bigl\{ A_{-k+l}  +   A_{-k-l}  \bigr\} B_{k} dk \\
% &= \int_{\Bbb R^2} \bigl\{ A_{k^\prime+l}  +   A_{k^\prime-l}  \bigr\} B_{-k^\prime} dk^\prime \\
% &= \int_{\Bbb R^2} \bigl\{ A_{k^\prime-l} + A_{k^\prime+l}  \bigr\} B_{k^\prime}^* dk^\prime
% \end{align*}
\end{proof}



%%%%%%%%%%%% lemma
The following lemma is equivalent to the so-called Convolution Theorem. We state it here for reference.
\begin{lemma} 
\label{conv}
If $A(x)$ and $B(x)$ are real scalar fields then  $\int_{\Bbb R^2} A_{k+l}  B^*_k \frac{dk}{2\pi}= \int_{\Bbb R^2} e^{-ix\cdot l} A(x)B(x)\frac{dx}{2\pi}$.
\end{lemma}

% \begin{proof} 
% \begin{align*}
% \int_{\Bbb R^2} A_{k+l}  B^*_k \frac{dk}{2\pi} 
% & = \int_{\Bbb R^2} \int_{\Bbb R^2} e^{-x\cdot (k+l)} A(x) B_{k}^*  \frac{dx}{2\pi}   \frac{dk}{2\pi} \\
% & = \int_{\Bbb R^2}e^{-ix\cdot l} A(x) \left[\int_{\Bbb R^2} e^{-x\cdot k} B_{k}^*     \frac{dk}{2\pi}\right] \frac{dx}{2\pi}\\
% & = \int_{\Bbb R^2}e^{-ix\cdot l} A(x) \left[\int_{\Bbb R^2} e^{x\cdot k} B_{k}     \frac{dk}{2\pi}\right]^* \frac{dx}{2\pi}\\
% &= \int_{\Bbb R^2} e^{-ix\cdot l} A(x)B^*(x)\frac{dx}{2\pi}\\
% &= \int_{\Bbb R^2} e^{-ix\cdot l} A(x)B(x)\frac{dx}{2\pi},\quad\text{since $B(x)$ is real.}
% \end{align*}
% \end{proof}




\end{document}

